% !TEX root =  master.tex
\chapter{Einleitung}
\nocite{*}

Das Testen ist ein wichtiger Teil eines jeden Softwareprojekts. Manuelles Testen kann dabei einen großen Zeit- und Arbeitsaufwand für ein Projekt bedeuten. Unter anderem aus diesem Grund werden Softwaretests automatisiert. Ziel dieser Arbeit ist es für den Testprozess eines Projektes zur Erstellung einer Anwendung, die \ac{KPI} berechnen und nutzerfreundlich anzeigen soll, Automatisierungsmöglichkeiten zu erkunden. Zu diesem Zweck werden für einen bestimmten Datenbankentest zwei Möglichkeiten zu dessen Automatisierung verglichen, ohne dass diese im vorliegenden Anwendungsfall eingesetzt werden müssen. Daraus soll eine Empfehlung resultieren, die das Projektteam bei der Entscheidung unterstützt, welche der beiden Varianten implementiert werden soll.
\newline


Zunächst werden detailliertere Informationen zu dem vorliegenden Projekt, sowie den in dieser Arbeit zu betrachtenden Tests gegeben. Darauf folgen grundlegende Themen zum Verständnis von Softwaretests, wie zum Beispiel der Testprozess oder einige Testarten. Im Anschluss werden Vorteile und Grenzen des automatisierten Testens beschrieben und zwei verschiedene Tools, die zur Automatisierung von Tests verwendet werden können, vorgestellt. Ein kurzer Überblick über einige während der Arbeit am vorliegenden Projekt verwendete Software schließen die theoretischen Grundlagen ab.
Anschließend werden detaillierte Informationen zum vorliegenden Projekt und dessen Testvorgängen näher beschrieben. Diese sind aufgeteilt in das Testen der Datenbank und das Testen der Anwendung. Für beide Fälle werden die bestehenden manuellen Tests, sowie die Konzepte für deren Automatisierung beleuchtet. Dabei ist eines davon ein bereits existierendes und das zweite ein vom Projektteam für diesen Anwendungsfall erstelltes System. Die Vor- und Nachteile der beiden Automatisierungsmöglichkeiten des Datenbankentests werden einzeln aufgezeigt und später miteinander verglichen. Zum Schluss erfolgt eine Empfehlung an das Projektteam, welche der beiden Varianten Anwendung finden sollte. Es wird erwartet, dass die vom Projektteam erstellte Variante in diesem Fall mehr Vorteile bietet.