% !TEX root =  master.tex
\chapter*{Abstract}

Ziel dieser Arbeit ist es für einen konkreten Anwendungsfall zwei verschiedene Automatisierungsmöglichkeiten eines Datenbankentests zu vergleichen, um einem Projektteam eine dieser Möglichkeiten zu empfehlen. Dabei handelt es sich bei der einen Variante um ein komplexes, bereits vorhandenes Test-Framework, das nicht genau auf diesen Anwendungsfall zugeschnitten ist, und bei der anderen um ein simples, vom Projektteam für diesen konkreten Anwendungsfall speziell entwickeltes. Da keine Möglichkeit eines Testlaufs der beiden bestanden, werden Vor- und Nachteile der zwei Varianten aufgezeigt und in Hinblick auf die Bedürfnisse des Anwendungsfalls gegenübergestellt. 
\newline

Die komplexere Variante übersteigt die simple nur im Punkt Zuverlässigkeit. Der Vergleich zeigt, dass für diesen Anwendungsfall die simplere Version dem Projekt durch das Zugeschnittensein auf denselben große Vorteile bietet und eine bessere Übersichtlichkeit über die Ergebnisse des Tests gewährt. Außerdem legt er nahe, dass die Zeitersparnis bei dieser Variante größer ist. Aus diesem Grund wird dem Projektteam empfohlen die einfachere Variante zu wählen.

