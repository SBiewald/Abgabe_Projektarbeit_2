% !TEX root =  master.tex
\chapter{Fazit und Ausblick}
\nocite{*}

Der im Rahmen dieser Arbeit durchgeführte Vergleich zweier Automatisierungsmöglichkeiten hat ein eindeutiges Ergebnis erbracht und führte zu einer begründeten Empfehlung an das Projektteam. Im Ergebnis wird in diesem Anwendungsfall die Verwendung der vom Autor entwickelten Variante empfohlen. Obwohl davon auszugehen ist, dass die 1. Variante durch Zuverlässigkeit punkten kann, sind für dieses Projekt die Zeiteinsparungen, die Einfachheit des Systems sowie die Übersichtlichkeit der Ergebnisse von zentraler Wichtigkeit und in allen drei Kategorien ist die zweite Variante der ersten überlegen. Gleichzeitig ist jedoch anzuraten den Mehraufwand anzunehmen, das für diesen Anwendungsfall empfohlene Skript entsprechenden Softwaretests zu unterziehen, um mögliche Fehleranfälligkeiten zu reduzieren. 
\newline


Die in dieser Arbeit verwendete Methode ist insofern kritisch zu betrachten, dass keine der beiden Varianten für den vorliegenden Anwendungsfall eingesetzt werden konnte. Daher liegen keine empirischen Daten zu Zeitersparnis und Zuverlässigkeit vor. Dennoch lassen die Ergebnisse des Vergleichs eindeutige Rückschlüsse darauf zu, welche der beiden Varianten für diesen Anwendungsfall größere Vorteile bringt.
\newline


Falls den Projektmitgliedern die Ergebnisse dieser Arbeit nicht ausreichen, um eine Entscheidung zu treffen, welches System implementiert werden soll, wäre eine empirische Untersuchung der beiden Möglichkeiten in der Live-Umgebung ein nächster Schritt.

